\chapter{Data Driven Surrogate Signal Extraction for Dynamic PET} \label{sec:data_driven_surrogate_signal_extraction_results}
    \newpage
    
    \longsection{Data Driven Surrogate Signal Extraction for Dynamic PET Introduction}{sec:data_driven_surrogate_signal_extraction_results_introduction}
        
    
    \longsection{PCA Data Driven Surrogate Signal Extraction Methods for Dynamic PET}{sec:pca_data_driven_surrogate_signal_extraction_methods_for_Dynamic_pet}
        This section investigates \gls{DD} methods to extract \gls{SS} from dynamic \gls{PET}. In particular this section will discuss methods using \gls{PCA} and compare them to similar methods, from the literature, which use \gls{SAM}.
        
        \subsection{Introduction} \label{sec:pca_data_driven_surrogate_signal_extraction_methods_for_dynamic_pet_introduction}
            As discussed in~\Fref{sec:respiratory_motion_in_pet} and \Fref{sec:impact_of_tof_on_respiratory_motion_model_estimation_using_nac_pet} respiratory motion correction is beneficial in positron emission tomography. Respiratory motion reduces image resolution in \gls{PET} by introducing blurring and mis-alignment artefacts~\boxcite{Nehmeh2008a}. Methods of motion correction include gated reconstruction, where the acquisition is binned based on a respiratory trace. Gating requires a \gls{SS} which reflects the position the patient is in the respiratory cycle over time. Methods to determine \glss{SS} include those which use an external devices, like the \gls{RPM}. However, a disadvantage of these methods are that they require the use of additional equipment and a change to clinical practise.
            
            Thus, \gls{DD} methods to extract \glss{SS} have become an alternative, these include \gls{PCA}~\boxcite{Thielemans2011}, for more detail see~\Fref{sec:data_driven}. However, data driven methods have the disadvantage that they are adversely affected by the tracer kinetics of a dynamic acquisition. To the best of our knowledge, current \gls{DD} \gls{SS} extraction methods fail for dynamic \gls{PET} data where the tracer kinetics, at the start of the acquisition, offer more variance than the respiratory motion does. Previously, a moving window based approach using \gls{SAM} was proposed~\boxcite{Schleyer2014}. This section seeks to evaluate several adaptions of the \gls{PCA} method through which it can be used with dynamic data. The methods explored in this section include using the use of a moving window, and re-use of the \glss{PC} from a later time point to estimate the \gls{SS} from earlier time points.
        
        \subsection{Methods} \label{sec:pca_data_driven_surrogate_signal_extraction_methods_for_dynamic_pet_methods}
            \begin{figure}
                \centering
                
                \includegraphics[width=0.6\linewidth]{figures/data_driven_surrogate_signal_extraction_results_1_flowchart.png}
                
                \captionsetup{singlelinecheck=false, justification=centering}
                \caption{A diagram showing the possible ways in which the method could be executed.}
                \label{fig:pca_data_driven_surrogate_signal_extraction_methods_for_dynamic_pet_methods_flowchart}
            \end{figure}
            
            The following subsections will address the method with respect to the diagram seen in~\Fref{fig:pca_data_driven_surrogate_signal_extraction_methods_for_dynamic_pet_methods_flowchart}.
            
            \subsection{Data Acquisition} \label{sec:pca_data_driven_surrogate_signal_extraction_methods_for_dynamic_pet_methods_data_acquisition}
                $21$ dynamic \gls{18F-FDG} acquisitions, with a \gls{FOV} covering the upper lung and heart, were acquired on a \gls{GE} Discovery $710$. Each acquisition lasted approximately \SI{20}{\minute} with the acquisition starting before injection of the radiotracer. \glss{SS} were acquired in parallel using a \gls{RPM}.
                
            \subsection{Data Preparation} \label{sec:pca_data_driven_surrogate_signal_extraction_methods_for_dynamic_pet_methods_data_preparation}
                Data were unlisted into low resolution sinograms, at a time interval of \SI{500}{\milli\second}, using the \gls{GE} PetToolbox following~\boxcite{Bertolli2018Data-DrivenTomography} resulting in sinograms with dimensions $95 x 16 x 47 x 11$ (radial positions $x$ angles $x$ transaxial plane $x$ \gls{TOF}). Data was preprocessed by first applying a Freeman-Tukey transformation~\boxcite{Freeman1950TransformationsRoot} before then applying a Yeo-Johnson power transformation~\boxcite{Yeo2000ASymmetry} this is in order to attempt to transform the Poisson distributed data to be more Gaussian-like. The Freeman-Tukey transformation was applied before the Yeo-Johnson power transformation as it was found that the Freeman-Tukey transformation made the Yeo-Johnson power transformation more reliable. Furthermore, it was also found that the Yeo-Johnson power transformation gave better results than the Freeman-Tukey transformation.
                
                The Freeman-Tukey transformation is defined as:

                \begin{equation} \label{eq:pca_data_driven_surrogate_signal_extraction_methods_for_dynamic_pet_methods_freeman_tukey}
                    S_g := \sqrt{S_p + 1} + \sqrt{S_p}
                \end{equation}

                \noindent where in~\Fref{eq:pca_data_driven_surrogate_signal_extraction_methods_for_dynamic_pet_methods_freeman_tukey} $S_g$ is the resultant, Gaussian distributed, sinogram of applying the Freeman-Tukey transformation to Poisson distributed sinogram $S_p$~\boxcite{Freeman1950TransformationsRoot}.
                
                The Yeo-Johnson power transformation is defined as:

                \begin{equation} \label{eq:pca_data_driven_surrogate_signal_extraction_methods_for_dynamic_pet_methods_yeo_johnson}
                    S_g := \begin{cases}
                                ((S_p + 1)^\lambda - 1) / \lambda                   & \quad \text{if } \lambda \neq 0 \text{, } S_p \geq 0 \\
                                \log(S_p + 1)                                       & \quad \text{if } \lambda = 0 \text{, } S_p \geq 0    \\
                                -[(-S_p + 1)^{(2 - \lambda)} - 1)] / (2 - \lambda)  & \quad \text{if } \lambda \neq 2 \text{, } S_p < 0    \\
                                -\log(-S_p + 1)                                     & \quad \text{if } \lambda = 2 \text{, } S_p < 0
                            \end{cases}
                \end{equation}

                \noindent where in~\Fref{eq:pca_data_driven_surrogate_signal_extraction_methods_for_dynamic_pet_methods_yeo_johnson} $S_g$ is the resultant, Gaussian distributed, sinogram of maximising the log likelihood of the Yeo-Johnson power transformation to Poisson distributed sinogram $S_p$~\boxcite{Yeo2000ASymmetry}.
            
            \subsection{Surrogate Signal Extraction} \label{sec:pca_data_driven_surrogate_signal_extraction_methods_for_dynamic_pet_methods_surrogate_signal_extraction}
                \subsubsection{Pre-Processing} \label{sec:pca_data_driven_surrogate_signal_extraction_methods_for_dynamic_pet_methods_pre_processing}
                    Following the diagram in~\Fref{fig:pca_data_driven_surrogate_signal_extraction_methods_for_dynamic_pet_methods_flowchart} the first major split in the method is between using \gls{PCA} or \gls{SAM}. \gls{SAM} is included, as mentioned previously in~\Fref{sec:pca_data_driven_surrogate_signal_extraction_methods_for_dynamic_pet_introduction}, as a comparison for the \gls{PCA} based methods. Thus, the \gls{SAM} based methods are less developed here to conform to literature, seen in~\boxcite{Schleyer2014}.
                    
                    Both the \gls{PCA} and \gls{SAM} based methods have the capacity for the input data to be Gaussian smoothed and further downsampled, different parameters are used for both methods. It has been found through experimentation that Gaussian smoothing can improve results, especially in the case of the \gls{SAM} methods. Further downsampling can be performed post smoothing to reduce memory usage, in this case Akima spline interpolation is used~\boxcite{Akima1970AProcedures}, however linear interpolation has been found to be satisfactory and additionally is less computationally expensive to calculate.
                        
                    Additionally, both types of method can be used with a mask, their implementation differs slightly however. In both cases the mask itself is defined as being true for any value in the sinogram above a predetermined threshold. In the case of the \gls{PCA} based methods, when calling \gls{PCA} values not in the mask are removed prior. However, in the \gls{SAM} case values not in the mask are set to be equal to zero.
                
                \subsubsection{Applying PCA} \label{sec:pca_data_driven_surrogate_signal_extraction_methods_for_dynamic_pet_methods_applying_pca}
                    Again, as can be seen from the diagram in~\Fref{fig:pca_data_driven_surrogate_signal_extraction_methods_for_dynamic_pet_methods_flowchart} there are two methods of applying either \gls{PCA} or \gls{SAM} which are common to both routes.
                    
                    \begin{itemize}
                        \item Firstly there is what shall be referred to as the static method. Here, either \gls{PCA} or \gls{SAM} is applied to the entire data set in one go as it would be if the data were from a static acquisition, hence the name.
                        
                        \item Secondly there is the moving window method. Here the data is split into a series of windows, where each subsequent window overlaps with the previous window by half its length. The size of each window is predetermined and selected experimentally. \gls{PCA} or \gls{SAM} is applied independently on each window and the results are averaged together. The windows overlap as the sign of the signal from each window is arbitrary and the overlapping allows for a common sign to be found by comparing the \gls{CC} of neighbouring windows and flipping windows where the \gls{CC} is negative. The motivation for attempting the moving window method is that small windows can be used at early time points, where the tracer kinetics are at their most severe, and longer windows can be used at later time points. Smaller windows may help to extract \gls{RM} where they are shorter than the frequency of the tracer kinetics.
                    \end{itemize}
                    
                    Additionally there is one method which is not possible to implement using \gls{SAM}.
                    
                    \begin{itemize}
                        \item The one \gls{PC} method splits the data into two channels, one which only contains late time point data where the tracer kinetics have diminished and one which contains all the data. The cutoff between early and late time point data is determined experimentally. \gls{PCA} is applied to the late time point data only. The \glss{PC} from the late time point data can then be taken and multiplied by the channel containing all of the data to give the weights contributing to that \gls{PC} for all time points. The motivation for attempting this method is that it was observed that \glss{PC} for late time point data didn't vary significantly when different windows were selected, however that was not true for early time point data. It could be hypothesised that because the \gls{RM} should be semi-consistent throughout the acquisition, then if a \gls{PC} is capturing the \gls{RM} at late time points then it should do the same at early time points.
                    \end{itemize}
                
                \subsubsection{Combining PCs} \label{sec:pca_data_driven_surrogate_signal_extraction_methods_for_dynamic_pet_methods_combining_pcs}
                    Furthermore, as an additional development of the previous methods a way to combine \glss{PC} was proposed. The motivation for this was that it could be observed that signals in the frequency window of \gls{RM} could be seen outside of the first few \glss{PC}. Additionally, a significant number of these had far less of a frequency contribution in the frequency window of the tracer kinetics. Although they may not be selected if only one \gls{PC} is used and determined by the greatest mean frequency contribution in the respiratory window, as in~\boxcite{Thielemans2011}, and~\boxcite{Bertolli2018Data-DrivenTomography}
                    
                    In order to achieve the above goals first the \gls{PSD} of each signal must be calculated using \gls{FFT}. These \gls{PSD} contain the frequency contribution of each signal between the frequencies of zero and one Hertz. Next frequency windows representing the content of information related to tracer kinetics, \gls{RM} and noise are defined. In this case they are defined as \SI{0.0}{\hertz} to \SI{0.1}{\hertz}, \SI{0.1}{\hertz} to \SI{0.4}{\hertz} and above \SI{0.4}{\hertz} respectively. The respiratory contribution within each windows is determined for each \gls{PC} by finding the mean magnitude within the windows. Ratios are then determined between the kinetic window and the respiratory window and the respiratory window and the noise window. Data are ordered such that the respiratory to kinetic window ratio is maximised, in other words the \gls{PC} which contributes the most respiratory information without contributing much kinetic information goes first. Any \gls{PC} where more noise is contributed than useful respiratory information is then thrown away.
                    
                    In the order determined in the previous step the \glss{PC} are iterated through and both summed and subtracted with a weighting based on their respiratory contribution and a new \gls{PSD} is found for both resulting signals. If one of the signals increases the respiratory frequency contribution more than the kinetic or noise contribution then it becomes the new best \gls{PC} and goes to the next iteration. \glss{PC} are both summed and subtracted to deal with the arbitrary sign problem mentioned earlier~\Fref{sec:pca_data_driven_surrogate_signal_extraction_methods_for_dynamic_pet_methods_applying_pca}. A similar method of combining signals can be seen in~\boxcite{Kesner2010AMethods}
                
                \subsubsection{Post-Processing} \label{sec:pca_data_driven_surrogate_signal_extraction_methods_for_dynamic_pet_methods_post_processing}
                    Regardless of method used there is still some effects of the tracer kinetics to be expected at early time points and noise throughout. Thus a method of parallel compression has been proposed to deal with the remaining tracer kinetics and a plethora of smoothing to deal with the noise.
                    
                    \begin{itemize}
                        \item Firstly there is what shall be referred to as parallel compression. This is a method borrowed from audio engineering whereby the signal is split into two channels, one has its dynamic range decimated and then they are summed back together. This has the effect of reducing large differences in the dynamic range of the signal without losing a lot of breath to breath variability. In order to achieve this here the signal is split into two channels and one channel is further split into a series of small moving windows. The channel comprised of moving windows now has the median value of each window subtracted and each window is divided by its \gls{MAD}. This channel then has its windows averaged back together before being combined with the unadulterated channel. The channels are combined following the head count signal. The head count signal is the gradient of the sum of the sinogram at each time point, this signal is normalised between zero and one and where it is larger more of the compressed signal is summed.
                        
                        \item Even though most of the large macro changes in intensity are dealt with by parallel compression some momentary spikes still make it through the process. Thus outliers are removed where they are outside a threshold of the quantile of the signal and new values are interpolated.
                        
                        \item Smoothing is applied first through the use of a bandpass filter before a Savitzky-Golay filter is applied. A Savitzky-Golay filter is a moving window polynomial filter. The parameters of both are determined through experimentation.
                    \end{itemize}
            
            \subsection{Evaluation} \label{sec:pca_data_driven_surrogate_signal_extraction_methods_for_dynamic_pet_methods_evaluation}
                \subsubsection{Correlation Coefficient} \label{sec:pca_data_driven_surrogate_signal_extraction_methods_for_dynamic_pet_methods_cross_correlation}
                    The \gls{CC} of each \gls{SS} between each method and the \gls{RPM}, for all acquisitions, has been calculated. Additionally, the early time point \glss{SS} for each method have been plotted against the \gls{RPM}.
            
        \subsection{Results} \label{sec:pca_data_driven_surrogate_signal_extraction_methods_for_dynamic_pet_results}
            \begin{figure}
                \centering
                
                \includegraphics[width=1.0\linewidth]{figures/data_driven_surrogate_signal_extraction_results_1_vanilla_surrogate_signal.png}
                
                \captionsetup{singlelinecheck=false, justification=centering}
                \caption{A visual comparison between the \gls{RPM} \gls{SS} and the \gls{SS} from the static \gls{PCA} method using only the first \gls{PC} for three patients between \SI{20}{\second} and \SI{160}{\second}.}
                \label{fig:pca_data_driven_surrogate_signal_extraction_methods_for_dynamic_pet_results_vanilla_surrogate_signal}
            \end{figure}
            
            \begin{figure}
                \centering
                
                \includegraphics[width=1.0\linewidth]{figures/data_driven_surrogate_signal_extraction_results_1_combined_surrogate_signal.png}
                
                \captionsetup{singlelinecheck=false, justification=centering}
                \caption{A visual comparison between the \gls{RPM} \gls{SS} and the \gls{SS} from the static \gls{PCA} method by combining the first $20$ \glss{PC} for three patients between \SI{20}{\second} and \SI{160}{\second}.}
                \label{fig:pca_data_driven_surrogate_signal_extraction_methods_for_dynamic_pet_results_combined_surrogate_signal}
            \end{figure}
            
            \begin{figure}
                \centering
                
                \includegraphics[width=1.0\linewidth]{figures/data_driven_surrogate_signal_extraction_results_1_box_plot.png}
                
                \captionsetup{singlelinecheck=false, justification=centering}
                \caption{A box plot of the \gls{CC} between static \gls{PCA} method both when using only the first \gls{PC} and by combining the first $20$ \glss{PC} for all patients between \SI{20}{\second} and \SI{160}{\second} and for the entire acquisition.}
                \label{fig:pca_data_driven_surrogate_signal_extraction_methods_for_dynamic_pet_results_box_plot}
            \end{figure}
            
            A plot of the \gls{SS} for static \gls{PCA} using only one \gls{PC} for three patients at early time points can be seen in~\Fref{fig:pca_data_driven_surrogate_signal_extraction_methods_for_dynamic_pet_results_vanilla_surrogate_signal}, it can be observed in this example that only using the static \gls{PCA} method and one \gls{PC} does not give satisfactory results at early time points. A plot of the \gls{SS} for static \gls{PCA} using a combination of the first $20$ \glss{PC} for three patients at early time points can be seen in~\Fref{fig:pca_data_driven_surrogate_signal_extraction_methods_for_dynamic_pet_results_combined_surrogate_signal}, here for all patients from very early in the acquisition it can be seen that the method gives comparable results to the \gls{RPM}. A box plot of the \gls{CC} of the \gls{SS} for static \gls{PCA} using only one \gls{PC} compared to the \gls{RPM} and the \gls{CC} of the \gls{SS} for static \gls{PCA} using a combination of the first $20$ \glss{PC} compared to the \gls{RPM} for all patients at both early time points and all time points and  can be seen in~\Fref{fig:pca_data_driven_surrogate_signal_extraction_methods_for_dynamic_pet_results_box_plot}, here the improvement by incorporating multiple \glss{PC} is most apparent. It must be noted that for these results that no pre or post processing of the sinograms nor signal is used, as such further improvement is to be expected.
            
        \subsection{Discussion and Conclusion} \label{sec:pca_data_driven_surrogate_signal_extraction_methods_for_dynamic_pet_discussion_and_conclusion}
            Results from a visual comparison of early time point signals from both an \gls{RPM} and static \gls{PCA} indicates that while using only one \gls{PC} leads to poor results the method of combining \glss{PC} presented here has good promise, this is further reinforced when the \gls{CC} is examined.
            
            In the future, research will focus on further development of method, tuning the parameters of the additional methods to the point where they can be compared.
    
%    \longsection{Feasibility Study of Neural Network Based Data Driven Surrogate Signal Extraction Methods for Dynamic PET}{sec:feasibility_study_of_neural_network_based_data_driven_surrogate_signal_extraction_methods_for_dynamic_pet}
        
        
%        \subsection{Introduction} \label{sec:feasibility_study_of_neural_network_based_data_driven_surrogate_signal_extraction_methods_for_dynamic_pet_introduction}
            
        
%        \subsection{Methods} \label{sec:feasibility_study_of_neural_network_based_data_driven_surrogate_signal_extraction_methods_for_dynamic_pet_methods}
            
            
%        \subsection{Results} \label{sec:feasibility_study_of_neural_network_based_data_driven_surrogate_signal_extraction_methods_for_dynamic_pet_results}
            
            
%        \subsection{Discussion and Conclusion} \label{sec:feasibility_study_of_neural_network_based_data_driven_surrogate_signal_extraction_methods_for_dynamic_pet_discussion_and_conclusion}
            
