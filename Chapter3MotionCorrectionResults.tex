\chapter{Impact of TOF on Respiratory Motion Model Estimation Using NAC PET} \label{sec:impact_of_tof_on_respiratory_motion_model_estimation_using_nac_pet"}
    
    
    \longsection{Impact of TOF on Respiratory Motion Model Estimation Using Pre-Gated No Intra-Cycle Motion NAC PET}{sec:impact_of_tof_on_respiratory_motion_model_estimation_using_pre_gated_no_intra_cycle_motion_nac_pet}
        This section investigates the possibility of using \glsing{MM} for respiratory \gls{MC} in \gls{PET}/\gls{CT}, and in particular whether incorporating \gls{TOF} information increases the accuracy of the \glss{MM} derived from \glsed{NAC} reconstructed images.
        
        \subsection{Introduction} \label{sec:impact_of_tof_on_respiratory_motion_model_estimation_using_pre_gated_no_intra_cycle_motion_nac_pet_introduction}
        \gls{RM} reduces image quality in \gls{PET} by causing artefacts and loss of resolution in the thoracic region~\boxcite{Nehmeh2008a}. Many methods have been proposed to correct for \gls{RM}, usually involving registration between a reference volume and a set of volumes in different positions in the respiratory cycle obtained by gating~\boxcite{Oliveira2014}. However, such pair wise registration is sensitive to noise. It also does not allow prediction of the respiratory state for data not used to estimate the motion, for instance, to be used for real time \gls{MC}. \gls{SS} driven \glss{MM} attempt to overcome these deficiencies by relating the motion in the data to a number of \gls{SS} values~\boxcite{McClelland2013}. The model can output a transformation or \gls{DVF} for every value of a \gls{SS}. \glss{MM} are fit on a series of either time or gating based volumes. This is as discussed in~\Fref{sec:motion_modelling}

        To avoid mis-registration due to attenuation mismatches, most existing methods rely on pair wise registration of \glsed{NAC} \gls{PET} volumes. The benefits of using \glsed{AC} \gls{PET} for \gls{IR} are unclear. If images are reconstructed using a static \gls{Mu-Map}, then artefacts caused by the misalignment between the activity distribution and the \gls{Mu-Map} would hamper \gls{IR}. It could therefore be advantageous to estimate motion on \gls{NAC} images~\boxcite{LungMotionDiaphragmBaiBib, Kalantari2017AttenuationRegistration:, Dawood2008RespiratoryAlgorithms, Dawood2006LungImages}. However, this is a challenging problem due to the low contrast and high noise of these volumes. Contrast may be too low to fit an accurate \gls{MM} and artefacts associated with the mismatch between the acquisition data and \gls{Mu-Map} could also obscure the underlying motion.
        
        In the absence of \gls{TOF}, there is no information on the activity position along the \gls{LOR} and \glsed{NAC} reconstructions have high intensity near the surface and low contrast in the internal part of the body. In \gls{TOF}, the time information constrains the activity position along the \gls{LOR} changing the nature and extent of the artefacts associated with \glsed{NAC} \gls{PET} as well as changing noise properties~\boxcite{Ter-Pogossian1981}.
        
        The aim of this section is to investigate whether \gls{TOF} can sufficiently increase the contrast and lower the noise of \glsed{NAC} images to facilitate the calculation of accurate \glss{MM}.
        
        \subsection{Methods} \label{sec:impact_of_tof_on_respiratory_motion_model_estimation_using_pre_gated_no_intra_cycle_motion_nac_pet_methods}
            \subsubsection{XCAT Image Generation} \label{sec:impact_of_tof_on_respiratory_motion_model_estimation_using_pre_gated_no_intra_cycle_motion_nac_pet_methods_xcat_image_generation}
                \gls{XCAT}~\boxcite{Segars2010} was used to generate six volumes over a linear \SI{5.0}{\second} second breathing cycle, as discussed in~\Fref{sec:respiratory_correspondence_model}, with one volume at full expiration at the beginning of the cycle and one volume at full expiration at the end of the cycle and using the default \gls{XCAT} settings for the extent of \gls{AP} and \gls{SI} motion. The default \gls{XCAT} settings being a, peak to peak, \SI{2.0}{\centi\metre} linear \gls{RM} displacement over \SI{5.0}{\second}. Activity concentrations were derived from a static \gls{F-FDG} patient scan. The \gls{FOV} included the base of the lungs, diaphragm and the top of the liver with a \SI{40.0}{\milli\metre} diameter spherical lesion placed in the right lung.
            
            \subsubsection{PET Data Simulation} \label{sec:impact_of_tof_on_respiratory_motion_model_estimation_using_pre_gated_no_intra_cycle_motion_nac_pet_methods_pet_data_simulation}
                \gls{PET} acquisitions were simulated using \gls{STIR}~\boxcite{Thielemans2012},~\boxcite{Efthimiou2018} through \gls{SIRF}~\boxcite{Ovtchinnikov2017} to forward project the input data to sinograms using the geometry of a \gls{GE} Discovery 690/710 and, where relevant, a \gls{TOF} resolution of \SI{375.0}{\pico\second} similar to the \gls{GE} Signa \gls{PET}/\gls{MR} (using \gls{TOF} mashing to reduce computation time resulting in $13$ \gls{TOF} time bins of size \SI{376.5}{\pico\second}). Attenuation was included in the simulation using the relevant \gls{Mu-Map} generated by \gls{XCAT}. Scatter and randoms were not taken into account in the simulation. Poisson noise realisations were generated to simulate an acquisition as if it had been gated into six bins over an acquisition of \SI{120}{\second}, emulating a standard single bed position acquisition. 
            
            \subsubsection{Image Reconstruction} \label{sec:impact_of_tof_on_respiratory_motion_model_estimation_using_pre_gated_no_intra_cycle_motion_nac_pet_methods_image_reconstruction}
                Data were reconstructed without \gls{AC} using \gls{OSEM} with two full iterations and $24$ subsets~\boxcite{Hudson1994}. Volumes were post filtered using a Gaussian blurring with a kernel size of \SI{6.4}{\milli\metre} \gls{FWHM}.
            
            \subsubsection{Motion Model Estimation} \label{sec:impact_of_tof_on_respiratory_motion_model_estimation_using_pre_gated_no_intra_cycle_motion_nac_pet_methods_motion_model_estimation}
                \gls{3D} \gls{BS}s were used to model spatial deformations with the corresponding warping operation denoted as $\mathbf{W}(\mathbf{\alpha}_t)$, with $\mathbf{\alpha}_t$ a vector with \gls{BS} coefficients at time $t$. The breathing \glss{SS} $\mathbf{s}$ contained two components: the \gls{AP} and \gls{SI} motion signals used by \gls{XCAT}. Following~\boxcite{McClelland2017} a direct correspondence \gls{MM} was used where the \gls{BS} coefficients at time $t$ are expressed as a linear combination of the two \gls{SS}, $s_{1,t}$ and $s_{2,t}$:
            
                \begin{equation}  \label{eq:b_spline_coefficients}
                    \forall t \in [[1,n_t]], \alpha_{k,t} := R_{1,k} s_{1,t} + R_{2,k} s_{2,t} + R_{3,k}
                \end{equation}
                
                \noindent where $\alpha_{k,t}$ is the \gls{3D} \gls{BS} coefficient for node $k$ at time point $t$, and $R_{i,k}$ are the model parameters, this is as discussed in~\Fref{sec:motion_modelling}.
            
                A generalised framework unifying registration and \gls{MM} estimation, NiftyRegResp, was used to estimate the \gls{RCM}. \glss{RCM} being the models fit on the acquired \gls{SS} and \glss{DVF} which, as a function can, take in a \gls{SS} value and return the \gls{DVF} which can be used to warp a volume to a reference position. Here, using \gls{SSD} as the objective function~\boxcite{McClelland2017}.
                
            \subsubsection{Evaluation} \label{sec:impact_of_tof_on_respiratory_motion_model_estimation_using_pre_gated_no_intra_cycle_motion_nac_pet_methods_evaluation}
                Three \glss{RCM} were compared: calculated from the \gls{PET} \gls{XCAT} volumes (gold standard), \gls{NTOF} \glsed{NAC} reconstructions and \gls{TOF} \glsed{NAC} reconstructions. To test the accuracy of the \glss{RCM}, the three models were used to warp the \gls{PET} volume generated by \gls{XCAT} at the mean breathing position, to the position at each gate. The mean breathin position \gls{XCAT} volume was generated by calculating the mean \gls{SS} value and using this as an input to \gls{XCAT}. A mean position volume was used as the \gls{RCM} was fit with this as the reference position for the \gls{SS}, as discussed in~\Fref{sec:motion_modelling}. These estimated volumes were then compared to the original \gls{XCAT} input volumes. Difference volumes were obtained by subtracting the original \gls{XCAT} volume $\mathbf{f}_t$ and warped volumes $\mathbf{W}(\alpha_t) \mathbf{f}_\mathrm{ref}$ at the same gate. \gls{MAPE} were computed from these difference images. \gls{MAPE} is expressed as:
                
                \begin{equation}  \label{eq:mape}
                   % M := \frac{\frac{\sum_{n}^{1}\abs{g - e}}{n}}{\frac{\sum_{n}^{1}g}{n}} \times 100
                   M := \frac{\frac{1}{n}\sum_{n}\mid e_n - g_n \mid}{\frac{1}{n}\sum_{n}g_n} \times 100
                \end{equation}
                
                \noindent where $n$ is the number of volumes, $e_n$ are the estimated volumes and $g_n$ are the \gls{GT} volumes.
                
                In addition, the \gls{COM} of the lesion was also tracked over the six gates, by warping a volume only including the lesion in the reference position as above, and then computing the \gls{COM}.
            
        \subsection{Results} \label{sec:impact_of_tof_on_respiratory_motion_model_estimation_using_pre_gated_no_intra_cycle_motion_nac_pet_results}
            \begin{figure}
                \centering
                
                \includegraphics[width=1.0\linewidth]{figures/result_1_output.png}
                
                \captionsetup{singlelinecheck=false, justification=raggedright}
                \caption{All volumes correspond to end inhalation. First row from left to right: \gls{XCAT} \gls{PET} data, \glsed{NAC} \gls{NTOF} reconstructed data and \glsed{NAC} \gls{TOF} reconstructed data. Second row: \glss{RCM} applied to mean position \gls{XCAT} data with \glss{RCM} derived from \gls{XCAT} \gls{PET} data (left), \glsed{NAC} \gls{NTOF} (middle) and \glsed{NAC} \gls{TOF} (right) volumes. Colour map ranges are consistent for all images on this row. The third row from left to right:  difference between the estimated volumes from the second row with the \gls{XCAT} end inhalation volume. Colour map ranges are consistent for all images on this row.} \label{fig:impact_of_tof_on_respiratory_motion_model_estimation_using_pre_gated_no_intra_cycle_motion_nac_pet_results_output}
            \end{figure}
            
            \begin{table}
                \centering
                
                \captionsetup{singlelinecheck=false, justification=raggedright}
                \caption{Comparison of the \gls{MAPE} between the \gls{GT} data and the volumes estimated from the \gls{XCAT} based \glss{RCM}, the volumes estimated from the \glsed{NAC} \gls{NTOF} based \gls{RCM} and the volumes estimated from the \glsed{NAC} \gls{TOF} based \gls{RCM}.}
                
                \resizebox*{0.75\linewidth}{!}
                {
                    \begin{tabular}{||c|ccc||}
                        \hline
                        \textbf{\gls{MAPE}} & \textbf{XCAT} & \textbf{\gls{NTOF}} & \textbf{\gls{TOF}} \\
                        \hline
                        \textbf{$1$} & $1.95$ & $8.35$ & $4.18$ \\
                        \textbf{$2$} & $1.59$ & $1.61$ & $1.84$ \\
                        \textbf{$3$} & $2.06$ & $9.91$ & $5.23$ \\
                        \textbf{$4$} & $1.97$ & $6.15$ & $3.68$ \\
                        \textbf{$5$} & $1.65$ & $4.45$ & $2.52$ \\
                        \textbf{$6$} & $1.95$ & $8.35$ & $4.18$ \\
                        \hline
                        \textbf{Mean} & $1.86$ & $6.47$ & $3.60$ \\
                        \hline
                    \end{tabular}
                } \label{tab:impact_of_tof_on_respiratory_motion_model_estimation_using_pre_gated_no_intra_cycle_motion_nac_pet_results_mape}
            \end{table}
            
            \begin{figure}
                \centering
                
                \includegraphics[width=1.0\linewidth]{figures/result_1_TOF.png}
                
                \captionsetup{singlelinecheck=false, justification=raggedright}
                \caption{The path of the \gls{COM} of the lesion, in voxel indices. Horizontal (respectively vertical) axis corresponds to motion in the \gls{AP} (respectively \gls{SI}) direction over the six gates. Different curves denote \gls{COM} displacement for \gls{GT} data, the estimated data from the \gls{XCAT} based \gls{RCM}, the estimated data from the \glsed{NAC} \gls{NTOF} based \gls{RCM} and the estimated data from the \glsed{NAC} \gls{TOF} based \gls{RCM}.} \label{fig:impact_of_tof_on_respiratory_motion_model_estimation_using_pre_gated_no_intra_cycle_motion_nac_pet_results_com_graph}
            \end{figure}
            
             The reconstructed data, estimated volumes and difference can be seen in~\Fref{fig:impact_of_tof_on_respiratory_motion_model_estimation_using_pre_gated_no_intra_cycle_motion_nac_pet_results_output} and \gls{MAPE} are in~\Fref{tab:impact_of_tof_on_respiratory_motion_model_estimation_using_pre_gated_no_intra_cycle_motion_nac_pet_results_mape}. The mean \gls{MAPE} was found to be lower for the \glsed{NAC} \gls{TOF} data than for the \glsed{NAC} \gls{NTOF}.
            
             \gls{COM} results can be seen in~\Fref{fig:impact_of_tof_on_respiratory_motion_model_estimation_using_pre_gated_no_intra_cycle_motion_nac_pet_results_com_graph}. The path of the \glsed{NAC} \gls{TOF} data follows the \gls{GT} path much closer than the \glsed{NAC} \gls{NTOF} data, and is quite close to the gold standard \gls{XCAT}-derived motion.
            
        \subsection{Discussion and Conclusion} \label{sec:impact_of_tof_on_respiratory_motion_model_estimation_using_pre_gated_no_intra_cycle_motion_nac_pet_discussion_and_conclusion}
            \glss{MM} derived from \glsed{NAC} \gls{TOF} volumes were found to be more robust than when using \glsed{NAC} \gls{NTOF}, both visually and when comparing \gls{MAPE} and \gls{COM}. This was noticeable for the lung lesion in the thoracic cavity but also for other parts of the anatomy such as the liver. This is likely due to the improved image contrast of \glsed{NAC} \gls{TOF} images.

            In the future, research will focus on investigating the robustness of the \gls{MM} estimation to different noise levels, acquisition duration and size of lesion.
    
    \longsection{Extension of NAC TOF PET Motion Model Estimation to Inter and Intra-Respiratory Cycle Variation Using Respiratory Gating}{sec:extension_of_nac_tof_pet_motion_modelling_to_inter_and_intra_respiratory_cycle_variation_using_respiratory_gating}
        
        
        \subsection{Introduction} \label{sec:extension_of_nac_tof_pet_motion_modelling_to_inter_and_intra_respiratory_cycle_variation_using_respiratory_gating_introduction}
        
        \subsection{Methods} \label{sec:extension_of_nac_tof_pet_motion_modelling_to_inter_and_intra_respiratory_cycle_variation_using_respiratory_gating_methods}
            
            
        \subsection{Results} \label{sec:extension_of_nac_tof_pet_motion_modelling_to_inter_and_intra_respiratory_cycle_variation_using_respiratory_gating_results}
            
            
        \subsection{Discussion and Conclusion} \label{sec:extension_of_nac_tof_pet_motion_modelling_to_inter_and_intra_respiratory_cycle_variation_using_respiratory_gating_discussion_and_conclusion}
            
            