\chapter{Introduction} \label{sec:introduction}
    
    
    \longsection{Motivation}{sec:motivation}
        \gls{PET}/\gls{CT} is a common medical imaging modality which combines \gls{PET}, a functional imaging modality used to capture images of the internal metabolic processes of a subject, with an X-ray based anatomical imaging modality called \gls{CT}. Every year in the UK just over $150,000$ \gls{PET}/\gls{CT} scans are performed, the majority of these scans are used in the diagnosis and treatment of cancer~\boxcite{NHSEngland2019Diagnostic2018/19}. \gls{PET} scans can take upwards of a few minutes to complete (usually between $1.5$ and $2.5$m), a single fast \gls{CT} scan (usually lasting between $5$ and $10$s) is used to correct for the attenuation of of the signal of the radioactive tracer by the matter of the patient.
        
        Simply, \gls{PET} works by quantifying the distribution of a radioactive tracer, which is injected into the patient. Quantification works by relating a \gls{LOR} to $2$ opposing gamma photons detected using rings of detectors. These opposing gamma photons are through the annihilation of positrons from the (decay of the) radioactive tracer with electrons. The measurement of these opposing gamma photons gives a line upon which the annihilation could have taken place. Image reconstruction attempts to go from this measurement space to image space reflecting the function of the radioactive tracer in the patient, this is an inverse problem going from measurements to the cause of an event.
        
        The \gls{CT} and \gls{PET} acquisitions of a combined \gls{PET}/\gls{CT} scan take place temporally apart from one another by a few minutes. Any movement during or between the acquisitions of a \gls{PET}/\gls{CT} scan will lead to the occurrence of artefacts and blurring, or the reduction of resolution, of the final image. Inter-acquisition artefacts are caused by a misalignment of the \gls{CT} and \gls{PET} volumes leading to attenuation being corrected for where it should not and vice versa. For intra-acquisition blurring, these artefacts are formed in a similar way to how blurring may appear during a long exposure photograph, where counts from one specific location are spread about amongst multiple voxels in the final image or volume. These errors lead to difficulty in the detection and location of lesions, for instance. Some sources of movement include: limb or head motion, cardiac motion, and \gls{RM} from patient breathing.
        
        Current common clinical practice is still to proceed with non \gls{MCIR} of \gls{PET}/\gls{CT} or to forgo post reconstruction \gls{MC}. This is because of the, usually, high computational expense and the distrust of evaluation techniques used to prove the effectiveness of \gls{MC} algorithms. New methods usually take a while to see clinical adoption due to the severe consequences of either a false positive or false negative diagnosis.
        
        Previous \gls{MC} solutions have mainly focused on binning \gls{PET} data into separate volumes where the difference between the volumes being binned together is low, co-registering these gated volumes and then summing the result together. \gls{PET} data is usually binned into a histogram based on a \gls{SS}, which reflects the position that the patient was in during that part of the acquisition. However, if a single \gls{CT} is used for attenuation correction, the misalignment problem may still exist as it is unlikely and not guaranteed for the position of this \gls{CT} to match any one \gls{PET} bin, never mind all of them.
        
        This \gls{SS} can be derived from a multitude of sources including from an external device, which mechanically measures the position of the patient, or through a \gls{DD} method, where the value of the \gls{SS} is derived solely from the data of the acquisition. External equipment for estimation of the \gls{SS} are not desirable due to the large impact on clinical workflow from having to affix the external device to the patient. Additionally, external device based \gls{SS} extraction methods are difficult to roll out on a large scale due to the equipment in question having to be purchased and shipped physically to each user.
        
        One way to resolve the issue of having none spatially matching \gls{CT} and \gls{PET} volumes is to to use Cine or \gls{4D} \gls{CT}. These types of \gls{CT} acquisition are taken continuously throughout the respiratory cycle, for instance, and thus have matching data for each position in the respiratory cycle. However, this requires a higher dose to the patient. This type of data can be replayed sequentially like the frames of a video hence the Cine or cinema and \gls{4D} names. Although,still this \gls{CT} data would not be simultaneously acquired with the \gls{PET} data so registration and misalignment issues will still be apparent but reduced. 
        
        This work will specifically focus on correcting for \gls{RM}, some reasons for this include:
        
        \begin{itemize}
            \item Limb and head motion is mostly comprised of \gls{RD}s. This means that while the object may, for instance, translate or rotate, the space between the points contained within the object do not change. This is in comparison to \gls{NRD}s where the distance between the points contained with the object can change. There is already a large amount of research in the field of \gls{RD} \gls{MC} and to a certain degree it could be considered a solved problem. % This could be simplified by thinking of a solid and a liquid, the solid can be transformed by pushing and pulling it but you cannot deform the solid itself, whereas with a liquid not only can it be pushed and pulled as a whole but the object it's self can be deformed. \gls{RM} is a \gls{NRD}.
            
            \item When a \gls{PET} acquisition is taken of the head, usually measures are taken to immobilise the patient. It is not possible to immobilise the respiratory or cardiac motion of a patient. While it may be true that patients can usually hold their breath during a short \gls{CT} acquisition it is not possible for them to hold their breath during a much longer \gls{PET} acquisition. Thus it is more necessary to correct for these types of inevitable motion.
            
            \item Cardiac motion is an autonomous cyclical motion which the patient doesn't have much control over. \gls{RM} in contrast is mostly autonomous, however patients can breath at vastly different rates and depths, sometimes leading to unpredictable respiration patterns. % and can even breath in non cyclical ways such as when coughing or sneezing, as sick people tend to do.
            Thus the impact of a method that can model non-cyclical unpredictable motion would be greater in the case of \gls{RM}.
        \end{itemize}
        
        These problems delay the use of advanced motion management strategies in the clinic. \gls{PET} data can already be gated using \gls{DD} techniques without need for external equipment, as discussed above. However, further improvements to the method are needed for the upper lung, as motion in this region is still significant. However, its magnitude is less than the diaphragm and is difficult to quantify. Moreover, a preliminary method to align a single breath hold \gls{CT} and respiratory gated \gls{PET} has been developed. However, this method is likely to be too slow for clinical applications and challenges may arise with larger magnitude or complex motion as this method still relies on co registration of each gate, which is sensitive to noise.
        
        \subsection{Objectives of this Work} \label{sec:objectives_of_this_work}
            The aim of this project is to formulate a method which produces \gls{PET}/\gls{CT} images which are corrected for \gls{RM} and are automatically aligned between \gls{PET} and \gls{CT} data. This will, most likely, be achieved through \gls{DD} gating and \gls{IR}. The performance, or susceptibility to noise, and ideally computation time could be improved by incorporating \gls{MM}s. This ideally will be achieved with minimal impact on the patient and clinical environment, without increased dose, without increasing scanning time. Ideally the work flow of the method is that of one which is transparent to both the patient and the clinicians, increasing the likelihood of clinical adoption. Evaluation will be performed on simulated and patient data with a comparison to current academic and industry methods.
    
    \longsection{Overview of this Thesis}{sec:overview_of_this_thesis}
        % the first chapter gives an overview of the physics underlying the work, the second chapter presents some initial resulst. we then give an overview of the future plan in chapter 4. chapter 5 gives the main conclusions
