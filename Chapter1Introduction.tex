\chapter{Introduction} \label{sec:introduction}
    \newpage
    
    \longsection{Introduction}{sec:introduction_introduction}
        An overview of the topics to be covered is given in this chapter in order to motivate the work performed, including; introducing the data used, the potential problems anticipated to be encountered, reasons why this problem was chosen to be tackled and also a small insight into previous work performed in a similar vein, with more detail given in subsequent chapters. This chapter will then move onto covering the intended objectives of this work before outlining the further structure of this report.
    
    \longsection{Motivation}{sec:motivation}
        %KT for whatever reason the first PET isn't spelled-out. I'd force it to do that here(if you can)
        %ACW done
        \gls{PET}/\gls{CT} is a hybrid medical imaging modality which combines \gls{PET}, a functional imaging technique used to capture data related to the internal metabolic processes of a subject, with an anatomical imaging modality called \gls{CT}. Every year in the UK just over $150,000$ \gls{PET}/\gls{CT} scans are performed, the majority of these scans are used in the diagnosis and treatment of cancer~\boxcite{NHSEngland2019Diagnostic2018/19}.
    
        %KT I'd consider moving this paragraph to Background, but as you wish
        %KT LOR is also not spelled-out. weird. but I wouldn't include LOR in the 2nd sentence, but a bit further down (where you say ``line'')
        %ACW done
        %you have ``XXX works'' in the first 2 sentences.
        %maybe ``PET works.. This is based on the detection of pairs of gamma photons using rings of detectors.''
        %ACW done, thank you
        \gls{PET} operates by quantifying the distribution of a radioactive tracer, which is administered to the patient. This is based on the detection of pairs of gamma photons using rings of detectors. These opposing gamma photons are created by the annihilation of positrons, produced by the (decay of the) radioactive tracer, with electrons. The measurement of these opposing gamma photons gives what is called a \gls{LOR} upon which the annihilation may have taken place. Image reconstruction attempts to find from this measurement space the image space which would reflect the distribution of the radioactive tracer in the patient. This is an inverse problem, meaning that from a number of measurements the underlying cause or system is found or approximated.
            
        The \gls{CT} and \gls{PET} acquisitions of a combined \gls{PET}/\gls{CT} scan take place temporally apart from one another by a few minutes. Any movement during or between the acquisition of \gls{CT} and \gls{PET} data will lead to the occurrence of artefacts and blurring, or the reduction of resolution, of the final image. A single bed position \gls{PET} scan can take upwards of a few minutes to complete (usually between \SI{90}{\second} and \SI{150}{\second}), clinical \gls{PET} acquisitions are usually made up of multiple bed positions. A single fast \gls{CT} scan (lasting approximately between \SI{5}{\second} and \SI{10}{\second}) is usually used to correct for the attenuation of the signal of the radioactive tracer by the matter of the patient. Inter-acquisition artefacts are caused by a misalignment of the \gls{CT} and \gls{PET} volumes leading to attenuation being corrected for where it should not and vice versa. The misalignment of the \gls{Mu-Map} can also introduce bias in the scatter correction. For intra-acquisition blurring, these artefacts are formed in a similar way to how blurring may appear during a long exposure photograph, where counts from one specific location are spread about amongst multiple voxels (pixels in the \gls{2D} photograph) in the final image or volume. These errors lead to difficulty in the detection and location of lesions, for instance. Some sources of movement include: Head motion, cardiac motion, and \gls{RM} from patient breathing.
    
        %KT first sentence reads a bit weird. I'd just say ``... is still to forego motion correction.'' (no need to distinguish between the 2 versions here)
        %ACW done, thank you
        %KT 2nd sentence. it's  not just ``distristin the evaluation technique'', but also the ``motion correction technique''. However, given the title of your work, I would add the additional set-up of equipment to monitor motion (it's true anyway!)
        %ACW partially done, I don't understand "additional set-up of equipment to monitor motion"
        In clinical practice, it is currently still often the case to forego \gls{MC}. This is because of the, usually, high computational expense and a lack of confidence in the \gls{MC} algorithms and any evaluation techniques used to prove the effectiveness of \gls{MC} algorithms.% New methods usually take a while to see clinical adoption due to the severe consequences of either a false positive, false negative or otherwise erroneous diagnosis or planning.
    
        %KT I don't understand what you want to say with ``where the difference between volumes...'' maybe you want to talk about ``intra-gate'' stuff? Otherwise, just cut.
        %ACW done
        Previous \gls{MC} solutions have mainly focused on binning \gls{PET} data into separate volumes where the intra-gate motion is low, co-registering these gated volumes and then summing the result together, this will be discussed further in~\Fref{sec:background}. \gls{PET} data is usually binned into a histogram based on a \gls{SS}, which reflects the position that the patient was in during that part of the acquisition. However, if a single \gls{CT} is used for attenuation correction, the misalignment problem may still exist as it is unlikely and not guaranteed for the position of this \gls{CT} to match any one \gls{PET} bin, never mind all of them.
        
        %KT somewhat strange to go back to mismatch after talking about the surrogate. I think swap the order of the 2 paragraphs
        %ACW done
        One way to resolve the issue of having no spatially matching \gls{CT} and \gls{PET} volumes is to use \gls{CCT} or \gls{4DCT}. These types of \gls{CT} acquisition are taken continuously with free breathing of the patient, and thus are more likely to have matching data for each position in the respiratory cycle.  This type of data can be replayed sequentially like the frames of a video, hence the \gls{CCT} or cinema and \gls{4D} names. However, this requires a higher dose to the patient. In addition, this \gls{CT} data would not be simultaneously acquired with the \gls{PET} data so registration and misalignment issues will still be apparent albeit reduced. 
        
        More recent research on \gls{MC} has begun to look into methods known as \gls{MM}. Here, rather than directly co-registering each individual volume, a \gls{RCM} is formed by fitting a model that relates a \gls{SS} and the data. This means that the \gls{RCM}, once fit, can be used like a function where values of a \gls{SS}, other than the ones used to fit the \gls{RCM}, can be input and the output would be a \gls{DVF}. These \glss{DVF} could then be used to \gls{MC} the data from which the \gls{SS} was obtained. An advantage of \gls{MM} is that it is less susceptible to noise when compared to co-registering. This is because the \gls{RCM} is fit over all of the data simultaneously and as such any outliers in the data have less of an effect on the overall result.
    
        %KT I'd change the last sentence to saying something about the  price only (after all, the logistics is  small compared to installing a PET  scanner..)
        %ACW done
        \glss{SS} can be derived from a multitude of sources including from an external device, which measures the position of the patient, or through a \gls{DD} method, where the value of the \gls{SS} is derived solely from the data of the acquisition. External equipment for estimation of the \gls{SS} are not desirable due to the large impact on clinical workflow from having to affix the external device to the patient. \gls{PET} data can already be gated using \gls{DD} techniques without need for external equipment, however this method struggles on dynamic acquisitions.% Additionally, external device based \gls{SS} extraction methods are difficult to roll out on a large scale due to the equipment in question having to be purchased and shipped physically to each user.
            
        This work will specifically focus on correcting for \gls{RM}, some reasons for this include:
            
        \begin{itemize}
            %KT I'd remove the ``limb'' here. There's really nothing whatsoever done on limb motion in PET, and arms don't move rigidly.
            %ACW done
            \item Head motion is mostly comprised of \glss{RD}. This means that while the object may, for instance, translate or rotate, the space between the points contained within the object do not change. This is in comparison to \glss{NRD} (where the distance between the points contained within the object can change). This concept could be simplified by thinking of a solid and a malleable object, the solid object can be transformed by translating every point of it uniformly but it cannot be deformed, whereas with the malleable object not only can it be translated as a whole but the object itself can be deformed. \gls{RM} is a \gls{NRD}. There is already a large amount of research in the field of \gls{RD} \gls{MC} and to a certain degree it could be considered a solved problem~\boxcite{Hill2001}.
                
            \item When a \gls{PET} acquisition is taken of the head, usually measures are taken to immobilise the patient. It is not possible to immobilise the respiratory or cardiac motion of a patient. While it may be true that patients can usually hold their breath during a short \gls{CT} acquisition it is not possible for them to hold their breath during a much longer \gls{PET} acquisition. Thus it is more necessary to correct for these types of inevitable motion.
    
            %KT change ``respiration patterns'' to ``motion patterns''?
            %ACW done
            \item Cardiac motion is an autonomous cyclical motion which the patient doesn't have conscious control over. \gls{RM}, in contrast, is mostly autonomous, however patients can breath at vastly different rates and depths, and can change these parameters over time or completely cease breathing for a period, sometimes leading to unpredictable motion patterns. % and can even breath in non cyclical ways such as when coughing or sneezing, as sick people tend to do.
            Thus the impact of a method that can model non-cyclical unpredictable motion would be greater in the case of \gls{RM}.
        \end{itemize}
    
        %KT this paragraph doesn't make too much sense. first sentence doesn't follow from the previous stuff. I don't think upper lung is a major issue (as motion is small). I'd just cut this paragraph.
        %ACW done
        %ACW i edited to make more sense and readded
        The problems, mentioned above, delay the use of advanced motion management strategies in the clinic. However, further improvements to the method are needed for small lesions at the boundary between the liver, diaphragm and lung, as motion in this region is significant and without \gls{MC} it is difficult to quantify or visually assess these lesions. Moreover, a preliminary method to align a single breath hold \gls{CT} and respiratory gated \gls{PET} has been developed. However, this method is likely to be too slow for clinical applications and challenges may arise with larger magnitude or complex motion as this method still relies on co-registration of each gate, which is sensitive to noise. The performance, or susceptibility to noise, and ideally computation time could be improved by incorporating \glss{MM}. 
            
        \subsection{Objectives of this Work} \label{sec:objectives_of_this_work}
            The aim of this project is to formulate a method which produces \gls{PET}/\gls{CT} images which are corrected for \gls{RM} and are automatically aligned between \gls{PET} and \gls{CT} data. This will be achieved through \gls{DDG} and \glss{MM}. %KT I'd replace IR with MM %ACW done
            This will be achieved with minimal impact on the patient and clinical environment, without increased dose and without increasing scanning time. Ideally the work flow of the method is that of one which is transparent to both the patient and the clinicians, increasing the likelihood of clinical adoption. Evaluation will be performed on simulated and patient data with a comparison to current academic and industry methods.
        
    \longsection{Overview of this Thesis}{sec:overview_of_this_thesis}
        % the first chapter gives an overview of the physics underlying the work, the second chapter presents some initial resulst. we then give an overview of the future plan in chapter 4. chapter 5 gives the main conclusions
        
        The overview of the physics underlying the work is given in the first chapter of this thesis, first an introduction to the physics of \gls{PET}, including; how the distribution of the radiotracer is quantified by the scanner, the different types of scans which can be performed in this manner, the problem of attenuation and why and how it is solved and the manner in which the quantified raw data is processed and output. This chapter then moves onto discussing the advantages of combined \gls{PET}/\gls{CT} scanners and the method through which the raw data from this are taken and processed in order to give an anatomical or functional image of the patient. Furthermore, the problem of motion is touched upon before introducing the ways in which it can be corrected for, including; \gls{IR}, respiratory gating and finally covering \gls{MM}.
        
        The second chapter of this thesis covers work which has been performed in the field of \gls{RMC}. Including; a section discussing the impact of \gls{TOF} data on the accuracy of \gls{MM} fit on \gls{NAC} \gls{PET} data and then used to \gls{MC} said data. This section includes a short introduction before describing the method, showing the results of the method and evaluating these results. The following section discusses a method for \gls{RMC} of \gls{PET} data using a single \gls{CT} \gls{Mu-Map} with deformation fields derived from \gls{NAC} \gls{TOF} \gls{PET} data, this work builds upon the work from the previous section. This section, again, includes a short introduction before describing the method, showing the results of the method and evaluating these results.
        
        The third chapter of this thesis introduces a tangent where work was performed to make the previous chapter applicable not only to static \gls{PET} data but also to dynamic \gls{PET} data. This chapter covers a method to acquire \glss{SS} from dynamic data.% and then how to apply \gls{MC} to dynamic data.
        Within this chapter again there is a short introduction before describing the method, showing the results of the method and evaluating these results.
        
        The final chapter of this thesis concludes the previous two chapters before highlighting and motivating the future work to be performed as well as proposing a timeline of how this would ideally be done.
