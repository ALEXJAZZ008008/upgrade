\chapter{Methods} \label{methods}
    \blindtext
    
    \longsection{Impact of TOF on Respiratory Motion Modelling using NAC PET}{impact_of_tof_on_respiratory_motion_modelling_using_nac_pet}
        \blindtext
        
        \subsection{Methods} \label{impact_of_tof_on_respiratory_motion_modelling_using_nac_pet_methods}
            \subsubsection{XCAT Image Generation} \label{impact_of_tof_on_respiratory_motion_modelling_using_nac_pet_methods_xcat_image_generation}
                \gls{XCAT}~\boxcite{Segars2010} was used to generate six volumes over a linear five second breathing cycle, with one volume at full expiration at the beginning of the cycle and one volume at full expiration at the end of the cycle and using settings for the extent of Anterior-Posterior (AP) and Superior-Inferior (SI) motion. Activity concentrations were derived from a static \gls{FDG} patient scan. The field of view included the base of the lungs, diaphragm and the top of the liver with a forty millimetre diameter spherical lesion placed in the right lung.
            
            \subsubsection{PET Data Simulation} \label{impact_of_tof_on_respiratory_motion_modelling_using_nac_pet_methods_pet_data_simulation}
                \gls{PET} acquisitions were simulated using \gls{STIR}~\boxcite{Thielemans2012},~\boxcite{Efthimiou2018} through \gls{SIRF}~\boxcite{ovtchinnikov2019SIRFSynergisticImage,ovtchinnikov_evgueni_2019_3548719} to forward project the input data to sinograms using the geometry of a GE Discovery 710 and, where relevant, a \gls{TOF} resolution of three hundred and seventy five picoseconds similar to the GE Signa \gls{PET}/\gls{MR} (using \gls{TOF} mashing to reduce computation time resulting in thirteen \gls{TOF} time bins of size three hundred and seventy six point five picoseconds). Attenuation was included in the simulation using the relevant \gls{mu-map} generated by \gls{XCAT}. Scatter and randoms were not taken into account in the simulation. Multiple noise realisations were generated to simulate an acquisition as if it had been gated into six bins over an acquisition of one hundred and twenty seconds, emulating a standard single bed position acquisition. 
            
            \subsubsection{Image Reconstruction} \label{impact_of_tof_on_respiratory_motion_modelling_using_nac_pet_methods_image_reconstruction}
                Data were reconstructed without attenuation correction using \gls{OSEM} with two full iterations and twenty four subsets~\boxcite{Hudson1994}. Volumes were post filtered using a Gaussian blurring with a kernel size of six point four millimeter \gls{FWHM}.
            
            \subsubsection{Motion Model Estimation} \label{impact_of_tof_on_respiratory_motion_modelling_using_nac_pet_methods_motion_model_estimation}
                3D B-splines were used to model spatial deformations with the corresponding warping operation denoted as $\mathbf{W}(\mathbf{\alpha}_t)$, with $\mathbf{\alpha}_t$ a vector with B-spline coefficients at time $t$. The breathing surrogate signals $\mathbf{s}$ contained two components, the \gls{AP} and \gls{SI} motion signals used by \gls{XCAT}.  Following~\boxcite{McClelland2017} a direct correspondence motion model was used where the B-spline coefficients at time $t$ are expressed as a linear combination of the two surrogate signals, $s_{1,t}$ and $s_{2,t}$:
            
                \begin{equation}
                    \forall t \in [[1,n_t]],\quad \alpha_{k,t} := R_{1,k} s_{1,t} + R_{2,k} s_{2,t} + R_{3,k}
                \end{equation}
                
                \noindent where $\alpha_{k,t}$ is the 3D B-spline coefficient for node $k$ at time point $t$, and $R_{i,k}$ are the model parameters.
            
                A generalised framework unifying image registration and respiratory motion models, NiftyRegResp, was used to estimate the \gls{RCM}s, which are the object that take in a surrogate signal value and a volume and warp the volume based on the value of the surrogate signal object, of the motion model, using \gls{SSD} as an objective function~\boxcite{McClelland2017}.
                
            \subsection{Evaluation} \label{impact_of_tof_on_respiratory_motion_modelling_using_nac_pet_methods_evaluation}
                We compared three \gls{RCM}s, calculated from the \gls{PET} \gls{XCAT} volumes (gold standard), \gls{NTOF} \gls{NAC} reconstructions and \gls{TOF} \gls{NAC} reconstructions. To test the accuracy of the \gls{RCM}s, the three models were used to warp the \gls{PET} volume generated by \gls{XCAT} at the mean breathing position to the position at each gate. These estimated volumes were then compared to the original \gls{XCAT} input volumes. Difference volumes were obtained by subtracting the original \gls{XCAT} volume $\mathbf{f}_t$ and warped volumes $\mathbf{W}(\alpha_t) \mathbf{f}_\mathrm{ref}$ at the same gate. \gls{MAPE} were computed from these difference images.
                
                In addition, the \gls{COM} of the lesion was also tracked over the six gates, by warping a volume only including the lesion in the reference position as above, and then computing the \gls{COM}.
            
        \subsection{Results} \label{impact_of_tof_on_respiratory_motion_modelling_using_nac_pet_results}
            \begin{figure*}
                \centering
                
                \includegraphics[width=1.0\linewidth]{figures/output.png}
                
                \captionsetup{singlelinecheck=false, justification=raggedright}
                \caption{All volumes correspond to end-inhalation. First row from left to right: \gls{XCAT} \gls{PET} data, \gls{NAC} \gls{NTOF} reconstructed data and \gls{NAC} \gls{TOF} reconstructed data. Second row: \gls{RCM} applied to mean position \gls{XCAT} data with \gls{RCM} derived from \gls{XCAT} \gls{PET} data (left), \gls{NAC} \gls{NTOF} (middle) and \gls{NAC} \gls{TOF} (right) volumes. Colour map ranges are consistent for all images on this row. The third row from left to right: The difference between the estimated volumes from the second row with the \gls{XCAT} end-inhalation volume. Colour map ranges are consistent for all images on this row.} \label{fig:output}
            \end{figure*}
            
            \begin{table}
                \centering
                
                \captionsetup{singlelinecheck=false, justification=raggedright}
                \caption{Comparison of the \gls{MAPE} between the ground truth data and the volumes estimated from the \gls{XCAT} based \gls{RCM}, the volumes estimated from the \gls{NAC} \gls{NTOF} based \gls{RCM} and the volumes estimated from the \gls{NAC} \gls{TOF} based \gls{RCM}.}
                
                \resizebox*{1.0\linewidth}{!}
                {
                    \begin{tabular}{||c|ccc||}
                        \hline
                        \textbf{\gls{MAPE}} & \textbf{XCAT} & \textbf{\gls{NTOF}} & \textbf{\gls{TOF}} \\
                        \hline
                        \textbf{$1$} & $1.95$ & $8.35$ & $4.18$ \\
                        \textbf{$2$} & $1.59$ & $1.61$ & $1.84$ \\
                        \textbf{$3$} & $2.06$ & $9.91$ & $5.23$ \\
                        \textbf{$4$} & $1.97$ & $6.15$ & $3.68$ \\
                        \textbf{$5$} & $1.65$ & $4.45$ & $2.52$ \\
                        \textbf{$6$} & $1.95$ & $8.35$ & $4.18$ \\
                        \hline
                        \textbf{Mean} & $1.86$ & $6.47$ & $3.60$ \\
                        \hline
                    \end{tabular}
                } \label{tab:mape}
            \end{table}
            
            \begin{figure}
                \centering
                
                \includegraphics[width=1.0\linewidth]{figures/TOF.png}
                
                \captionsetup{singlelinecheck=false, justification=raggedright}
                \caption{The path of the \gls{COM} of the lesion. Horizontal (respectively vertical) axis corresponds to motion in the \gls{AP} (respectively \gls{SI}) direction over the six gates. Different curves denote \gls{COM} displacement for  ground truth data, the estimated data from the \gls{XCAT} based \gls{RCM}, the estimated data from the \gls{NAC} \gls{NTOF} based \gls{RCM} and the estimated data from the \gls{NAC} \gls{TOF} based \gls{RCM}.} \label{fig:com_graph}
            \end{figure}
            
             The reconstructed data, estimated volumes and difference can be seen in Fig~\ref{fig:output} and \gls{MAPE} are in Table~\ref{tab:mape}. The mean \gls{MAPE} was found to be lower for the \gls{NAC} \gls{TOF} data than for the \gls{NAC} \gls{NTOF}.
            
             \gls{COM} results can be seen in Fig~\ref{fig:com_graph}. The path of the \gls{NAC} \gls{TOF} data follows the ground truth path much closer than the \gls{NAC} \gls{NTOF} data, and is quite close to the gold standard \gls{XCAT}-derived motion.
            
        \subsection{Discussion and Conclusion} \label{impact_of_tof_on_respiratory_motion_modelling_using_nac_pet_discussion_and_conclusion}
            Motion models derived from \gls{NAC} \gls{TOF} volumes were found to be more robust than when using \gls{NAC} \gls{NTOF}, both visually and when comparing \gls{MAPE} and \gls{COM}. This was noticeable for the lung lesion in the thoracic cavity but also for other parts of the anatomy such as the liver. This is likely due to the improved image contrast of \gls{NAC} \gls{TOF} images.

            In the future, research will focus on investigating the robustness of the motion model estimation to different noise levels, acquisition duration and size of lesion.
    
    \longsection{Impact of TOF on Respiratory Motion Modelling using NAC PET: an Extension to Inter and Intra Respiratory Cycle Variation}{impact_of_tof_on_respiratory_motion_modelling_using_nac_pet_an_extension_to_inter_and_intra_respiratory_cycle_variation}
        \blindtext
        
        \subsection{Methods} \label{impact_of_tof_on_respiratory_motion_modelling_using_nac_pet_an_extension_to_inter_and_intra_respiratory_cycle_variation_methods}
            \blindtext
            
        \subsection{Results} \label{impact_of_tof_on_respiratory_motion_modelling_using_nac_pet_an_extension_to_inter_and_intra_respiratory_cycle_variation_results}
            \blindtext
            
        \subsection{Discussion and Conclusion} \label{impact_of_tof_on_respiratory_motion_modelling_using_nac_pet_an_extension_to_inter_and_intra_respiratory_cycle_variation_discussion_and_conclusion}
            \blindtext
    
    \longsection{Extension of Static PCA Based Data Driven Surrogate Signal Extraction to Dynamic PET}{extension_of_static_pca_based_data_driven_surrogate_signal_extraction_to_dynamic_pet}
        \blindtext
        
        \subsection{Methods} \label{extension_of_static_pca_based_data_driven_surrogate_signal_extraction_to_dynamic_pet_methods}
            \blindtext
            
        \subsection{Results} \label{extension_of_static_pca_based_data_driven_surrogate_signal_extraction_to_dynamic_pet_results}
            \blindtext
            
        \subsection{Discussion and Conclusion} \label{extension_of_static_pca_based_data_driven_surrogate_signal_extraction_to_dynamic_pet_discussion_and_conclusion}
            \blindtext
    
    \longsection{Feasibility of Neural Network Based Data Driven Surrogate Signal Extraction Methods for Dynamic PET}{feasibility_of_neural_network_based_data_driven_surrogate_signal_extraction_methods_for_dynamic_pet}
        \blindtext
        
        \subsection{Methods} \label{feasibility_of_neural_network_based_data_driven_surrogate_signal_extraction_methods_for_dynamic_pet_methods}
            \blindtext
            
        \subsection{Results} \label{feasibility_of_neural_network_based_data_driven_surrogate_signal_extraction_methods_for_dynamic_pet_results}
            \blindtext
            
        \subsection{Discussion and Conclusion} \label{feasibility_of_neural_network_based_data_driven_surrogate_signal_extraction_methods_for_dynamic_pet_discussion_and_conclusion}
            \blindtext
