\chapter{Background} \label{background}
    \blindtext
    
    \longsection{PET}{pet}
        \blindtext
        % general intro
        
        \subsection{The Physics of PET} \label{the_physics_of_pet}
            % write something about radiotracer, functional imaging, radiounclide decays, and photon interaction with matter
            \gls{PET} is an example of a type of imaging known as functional imaging. It is functional as rather than capturing images of physical things, in the sense that they take physical presence or form that you can see or touch, it images the metabolic functions of a living thing. This metabolic function is, for instance, how blood flows through and into parts of the body and how glucose etc is absorbed into certain cells.
            
            The process through which a \gls{PET} scan takes place is as follows; firstly, a patient is injected with a solution called a radiotracer. A radiotracer is a biologically active molecule which has been labelled with a positron emitting radionuclide, the molecule will have been selected knowing that it will have significant uptake in the \gls{ROI}, in other words after perfusion it will appear more often within certain types of cells. The positron emitting radionuclide is selected because it has the capacity to be labelled to the molecule by replacing one of its groups, not all radionuclides can be labeled to all molecules. After injection the radiotracer perfuses through the circulatory system of the patient.
            
            Some examples of common radionuclides used in radiotracers include; fluorine eighteen, gallium sixty seven, gallium sixty eight, and rubidium eighty two. Radionuclides have a half life which is the amount of time that statistically the amount of activity of the radionuclide will reduce by half. For the examples above these half lifes are approximately; one hundred and ten minutes, sixty seven minutes and one point two minutes respectively. An example of the use of some of these radionuclides are; fluorine eighteen can be labeled to glucose molecules and as such is uptaken in regions which are doing a lot of 'work', gallium sixty seven and gallium sixty eight are often used to target \gls{PSMA} which can be used in the detection of prostate cancer and rubidium eighty two can be used to image the heart in a scan targeting myocardial perfusion.
            
            Radionucleides used in \gls{PET} undergo a type of decay called beta decay. This is where, because the atom is unstable, a proton in its nucleus converts into a neutron releasing a positron and a neutrino. The emitted positron moves with some decreasing velocity through the body of the patient, for a distance called positron range, before eventually colliding and annihilating with it's antiparticle the electron. This annihilation causes the emission of two five hundred and eleven kilo electron volt gamma photons at almost one hundred and eighty degrees apart from one another. The gamma photons are almost never exactly one hundred and eighty degrees apart as the positron will usually have a small velocity going into the annihilation which is conserved after the annihilation in a velocity which is applied to the gamma photons.
            
            \subsubsection{Attenuation} \label{attenuation}
                \blindtext
        
        \subsection{Data acquisition} \label{data_acquisition}
            \blindtext
            % write a general intro on the scanner structure, and the type of events you detect (true scatter randoms etc)
            
            \subsubsection{Photon Detection} \label{photon_detection}
                \blindtext
            
            \subsubsection{Coincidence Processing} \label{coincidence_processing}
                \blindtext
                % write stuff about coincidence processing , give some info on TOF (or separate section for it, i don't know)
            
            \subsubsection{Static and Dynamic Acquisition} \label{static_and_dynamic_acquisition}
                \blindtext
            
            \subsubsection{Time of Flight PET} \label{tof_pet}
                \blindtext
            
        \subsection{Attenuation Correction} \label{attenuation_correction}
            \blindtext
            
            \subsection{Combined PET/CT} \label{attenuation_correction}
                \blindtext
    
    \longsection{PET Image Reconstruction}{pet_image_reconstruction}
        \blindtext
        
        \subsection{Analytic Image Reconstruction} \label{analytic_image_reconstruction}
            \blindtext
            % mention briefely FBP ?
        
        \subsection{Iterative Image Reconstruction} \label{iterative_image_reconstruction}
            \blindtext
            
            \subsubsection{Objective Function} \label{iterative_image_reconstruction_objective_function}
                \blindtext
                
            \subsubsection{Optimiser} \label{iterative_image_reconstruction_optimiser}
                \blindtext
    
    \longsection{Respiratory Motion in PET}{respiratory_motion_in_pet}
        \blindtext
        % general intro
        
        \subsection{Respiratory Motion Artefacts} \label{respiratory_motion_artefacts}
            \blindtext
            % have a look at https://www.sciencedirect.com/science/article/pii/S0001299808000214?via%3Dihub
            
        \subsection{Respiratory Motion Challenges in Combined PET/CT Imaging} \label{respiratory_motion_challenges_in_combined_pet_ct_imaging}
            \blindtext
    
    \longsection{Motion Correction for PET}{motion_correction_for_pet}
        \blindtext
    
        \subsection{Image Registration} \label{image_registration}
            \blindtext
            
            \subsubsection{Objective Function} \label{image_registration_objective_function}
                \blindtext
                
            \subsubsection{Optimiser} \label{image_registration_optimiser}
                \blindtext
            
        \subsection{Respiratory Signal Detection} \label{respiratory_signal_detection}
            \blindtext
            
            \subsubsection{External Devices} \label{external_devices}
                \blindtext
                
            \subsubsection{Data Driven} \label{data_driven}
                \blindtext
                
        \subsection{Respiratory Gating} \label{respiratory_gating}
            \blindtext
            % brief intro
            
        \subsection{Motion Modelling} \label{motion_modelling}
            \blindtext
        
        \subsection{Applying Motion Correction} \label{applying_motion_correction}
            \blindtext
            
            \subsubsection{Post Image Reconstruction Motion Correction} \label{post_image_reconstruction_motion_correction}
                \blindtext
                
            \subsubsection{Iterative Motion Corrected Image Reconstruction} \label{iterative_motion_corrected_image_reconstruction}
                \blindtext
    
    \longsection{Machine Learning for PET}{machine_learning_for_pet}
        \blindtext
        
        \subsection{Machine Learning Concepts} \label{machine_learning_concepts}
            \blindtext
        
        \subsection{Machine Learning for PET Image Reconstruction} \label{machine_learning_for_pet_image_reconstruction}
            \blindtext
            %might want to mention this?
        
        \subsection{Machine Learning for Motion Correction} \label{machine_learning_for_motion_correction}
            \blindtext
