\chapter{Background} \label{background}
    \blindtext
    
    \longsection{PET}{pet}
        \blindtext
        % general intro
        
        \subsection{The Physics of PET} \label{the_physics_of_pet}
            % write something about radiotracer, functional imaging, radiounclide decays, and photon interaction with matter
            \gls{PET} is an example of a type of imaging known as functional imaging. It is functional as rather than capturing images of physical things, in the sense that they take physical presence or form that you can see or touch, it images the metabolic processes of a living thing. This metabolic function is, for instance, how blood flows through and into parts of the body (perfusion) or how glucose is transported to and metabolised by certain cells.
            
            The process through which a \gls{PET} scan takes place is as follows; firstly, a patient is injected with a chemical compound called a radiotracer which is a solution containing radionuclides. A radiotracer is a biologically active molecule which has been labelled with a positron emitting radionuclide. The molecule will have been selected knowing that it will have significant uptake in the \gls{ROI} depending on the target tissue. In other words, after perfusion (through the circulatory system) it will appear with a greater frequency within certain types of cells (post metabolistation). The positron emitting radionuclide is selected because it has the capacity to be labelled to the molecule by replacing one of its groups, not all radionuclides can be labeled to all molecules. % After injection the radiotracer perfuses through the circulatory system of the patient.
            
            Some examples of common radionuclides used in radiotracers include; fluorine $18$, gallium $67$, gallium $68$, and rubidium $82$. Radionuclides have a half life which is the amount of time that statistically the amount of activity of the radionuclide will reduce by half. For the examples above these half lifes are approximately; $110$ minutes, $67$ minutes and $1.1$ minutes respectively.
            
            An example of the use of some of these radionuclides are:
            
            \begin{itemize}
                \item Glucose molecules can be labeled with fluorine $18$ (called \gls{FDG}). Glucose is used by cells, though glycolosis, in the carbohydrate metabolisation process to produce \gls{ATP} to make energy available to a cell. When a cell requires more energy it also requires more glucose and as such uptaken of \gls{FDG} is increased in these regions. Fluorine $18$ collects, over time, intra-cellularly because fluorine $18$ labeled \gls{FDG} cannot be fully metabolised due to the group which is replaced by fluorine $18$ being missing and required for this process. Thus the distribution of fluorine $18$ is a good reflection of glucose metabolisation and uptake over time.
                
                \item Gallium $67$ and gallium $68$ are often used to target \gls{PSMA} which can be used in the detection of prostate cancer.
                
                \item Rubidium $82$ can be used to image the heart in a scan targeting myocardial perfusion.
            \end{itemize}
            
            Radionucleides used in \gls{PET} undergo a type of decay called $\beta-$decay. This is where, because the radionuclide is unstable (because it has a high number of protons), a proton in its nucleus converts into a neutron releasing a positron and a neutrino. The emitted positron travels with some decreasing velocity through the body of the patient, for a distance called positron range, before eventually colliding and annihilating with its antiparticle the electron. This annihilation causes the emission of two $511$ \gls{KeV} gamma photons at $180^{\circ}$ apart from one another. However, because the centre of mass of the positron-electron system is not always at rest at the moment of their annihilation, the pair of emitted photons present a certain degree of non-collinearity, according to the laws of conservation of energy and momentum. This means that the gamma photons are never exactly $180^{\circ}$ apart, because the positron will usually have a small kinetic energy going into the annihilation. This kinetic energy is conserved after the annihilation as a force in a different direction, usually, to that of the force from the annihilation itself.
            
            \subsubsection{Attenuation} \label{attenuation}
                \blindtext
        
        \subsection{Data acquisition} \label{data_acquisition}
            \blindtext
            % write a general intro on the scanner structure, and the type of events you detect (true scatter randoms etc)
            
            \subsubsection{Photon Detection} \label{photon_detection}
                \blindtext
            
            \subsubsection{Coincidence Processing} \label{coincidence_processing}
                \blindtext
                % write stuff about coincidence processing , give some info on TOF (or separate section for it, i don't know)
            
            \subsubsection{Static and Dynamic Acquisition} \label{static_and_dynamic_acquisition}
                \blindtext
            
            \subsubsection{Time of Flight PET} \label{tof_pet}
                \blindtext
            
        \subsection{Attenuation Correction} \label{attenuation_correction}
            \blindtext
            
            \subsection{Combined PET/CT} \label{combined_pet_ct}
                \blindtext
    
    \longsection{PET Image Reconstruction}{pet_image_reconstruction}
        \blindtext
        
        \subsection{Analytic Image Reconstruction} \label{analytic_image_reconstruction}
            \blindtext
            % mention briefely FBP ?
        
        \subsection{Iterative Image Reconstruction} \label{iterative_image_reconstruction}
            \blindtext
            
            \subsubsection{Objective Function} \label{iterative_image_reconstruction_objective_function}
                \blindtext
                
            \subsubsection{Optimiser} \label{iterative_image_reconstruction_optimiser}
                \blindtext
    
    \longsection{Respiratory Motion in PET}{respiratory_motion_in_pet}
        \blindtext
        % general intro
        
        \subsection{Respiratory Motion Artefacts} \label{respiratory_motion_artefacts}
            \blindtext
            % have a look at https://www.sciencedirect.com/science/article/pii/S0001299808000214?via%3Dihub
            
        \subsection{Respiratory Motion Challenges in Combined PET/CT Imaging} \label{respiratory_motion_challenges_in_combined_pet_ct_imaging}
            \blindtext
    
    \longsection{Motion Correction for PET}{motion_correction_for_pet}
        \blindtext
    
        \subsection{Image Registration} \label{image_registration}
            \blindtext
            
            \subsubsection{Objective Function} \label{image_registration_objective_function}
                \blindtext
                
            \subsubsection{Optimiser} \label{image_registration_optimiser}
                \blindtext
            
        \subsection{Respiratory Signal Detection} \label{respiratory_signal_detection}
            \blindtext
            
            \subsubsection{External Devices} \label{external_devices}
                \blindtext
                
            \subsubsection{Data Driven} \label{data_driven}
                \blindtext
                
        \subsection{Respiratory Gating} \label{respiratory_gating}
            \blindtext
            % brief intro
            
        \subsection{Motion Modelling} \label{motion_modelling}
            \blindtext
        
        \subsection{Applying Motion Correction} \label{applying_motion_correction}
            \blindtext
            
            \subsubsection{Post Image Reconstruction Motion Correction} \label{post_image_reconstruction_motion_correction}
                \blindtext
                
            \subsubsection{Iterative Motion Corrected Image Reconstruction} \label{iterative_motion_corrected_image_reconstruction}
                \blindtext
    
    \longsection{Machine Learning for PET}{machine_learning_for_pet}
        \blindtext
        
        \subsection{Machine Learning Concepts} \label{machine_learning_concepts}
            \blindtext
        
        \subsection{Machine Learning for PET Image Reconstruction} \label{machine_learning_for_pet_image_reconstruction}
            \blindtext
            %might want to mention this?
        
        \subsection{Machine Learning for Motion Correction} \label{machine_learning_for_motion_correction}
            \blindtext
