\chapter{Background} \label{background}
    \blindtext
    
    \longsection{PET}{pet}
        \blindtext
        % general intro
        
        \subsection{The Physics of PET} \label{the_physics_of_pet}
            % write something about radiotracer, functional imaging, radiounclide decays, and photon interaction with matter
            \gls{PET} is an example of a type of imaging known as functional imaging. It is functional as rather than capturing images of anatomy for instance the structure and density of bones, it images the metabolic processes of a living thing. This metabolic function is, for instance, how blood flows through and into parts of the body (perfusion) or how glucose is transported to and metabolised by certain cells.
            
            The process through which a \gls{PET} scan takes place is as follows; firstly, a patient is injected with a chemical compound called a radiotracer which is a solution containing radionuclides. A radiotracer is a biologically active molecule which has been labelled with a positron emitting radionuclide. The molecule will have been selected knowing that it will have significant uptake in the \gls{ROI} depending on the target tissue. In other words, after perfusion (through the circulatory system) it will appear with a greater frequency within certain types of cells (post metabolistation). The positron emitting radionuclide is selected because it has the capacity to be labelled to the molecule by replacing one of its groups, not all radionuclides can be labeled to all molecules. % After injection the radiotracer perfuses through the circulatory system of the patient.
            
            Some examples of common radionuclides used in radiotracers include; fluorine $18$, gallium $67$, gallium $68$, and rubidium $82$. The rate at which a radionuclide decays is measured in terms of half life. The half-life is defined as the amount of time for which the number of atoms of a radioactive material reduces by half. Half lives for various radioisotopes can range from a few microseconds to billions years. For the examples above these half lifes are approximately; $110$ minutes, $67$ minutes and $1.1$ minutes respectively~\boxcite{FDGGuidelines}.
            
            An example of the use of some of these radionuclides are:
            
            \begin{itemize}
                \item Glucose molecules can be labeled with fluorine $18$ (called \gls{FDG}). Glucose is used by cells, though glycolosis, in the carbohydrate metabolisation process to produce \gls{ATP} to make energy available to a cell. When a cell requires more energy it also requires more glucose and as such uptaken of \gls{FDG} is increased in these regions. Fluorine $18$ collects, over time, intra-cellularly because fluorine $18$ labeled \gls{FDG} cannot be fully metabolised due to the group which is replaced by fluorine $18$ being missing and required for this process. Thus the distribution of fluorine $18$ is a good reflection of glucose metabolisation and uptake over time. \gls{FDG} is by far the most commonly used radiotracer in \gls{PET}~\boxcite{WeissBook, FDGGuidelines}.
                
                \item Gallium $67$ and gallium $68$ are often used to target \gls{PSMA} which can be used in the detection of prostate cancer. \gls{PSMA} is a protein which is present in prostate cancer cells.
                
                \item Rubidium $82$ can be used to image the heart in a scan targeting myocardial perfusion~\boxcite{Selwyn1982}.
            \end{itemize}
            
            Radionucleides used in \gls{PET} undergo a type of decay called $\beta-$decay~\boxcite{conti_beta}. This is due to an instability of the radionuclide, given to an imbalance of neutrons to protons. As a consequence, a proton in its nucleus converts into a neutron releasing a positron and a neutrino. The emitted positron travels with some decreasing velocity, from sequential collisions, through the body of the patient, for a distance called positron range, before its final collision and annihilation with its antiparticle the electron~\boxcite{EvansPositronBib}. A \gls{PET} scanner thus does not image, directly, the emission of the positron but, in fact, more closely images the location of the annihilation.
            
            The annihilation causes the emission of two $511$ \gls{KeV} gamma photons at $180^{\circ}$ apart from one another. However, because the poistron-electron pair may not be at rest at the moment of their annihilation, the two emitted photons can show a certain degree of non-collinearity, according to the laws of conservation of energy and momentum. This means that the gamma photons are never exactly $180^{\circ}$ apart~\boxcite{pet_basic}.%, because the positron will usually have a small kinetic energy going into the annihilation. This kinetic energy is conserved after the annihilation as a force in a different direction, usually, to that of the force from the annihilation itself.
            
            The \gls{FOV} of the scanner is the area in which it can detect these gamma photons. Current clinical \gls{PET} scanners, usually, have a cylindrical \gls{FOV} with a length of between $15$ and $25$ centimetres and a diameter of between $50$ and $70$ centimetres~\boxcite{Pan2019}. In order to take acquire data over an area of more than the length of the scanner acquisitions can be taken over multiple positions and summed together. Alternatively, whole body \gls{PET} scanners are becoming more viable for research and have a high potential for clinical applications. However, the increased price and size constitute a limitation. Whole body \gls{PET} scanners have an axial \gls{FOV} which contains most of the patients body making multiple acquisitions less necessary while also increasing the sensitivity to detecting annihilations~\boxcite{Cherry2018}.
            
            \subsubsection{Static and Dynamic Acquisition} \label{static_and_dynamic_acquisition}
                There are two main types of \gls{PET} scan useful for determining separate processes, these types of scans and uses are:
                
                \begin{itemize}
                    \item The first and most common type of \gls{PET} scan is a static \gls{PET} scan. The patient is scanned only when the injected radiotracer has perfused through their body and eventually stabilised at a semi-consistant distribution. The time elapsed between injection and acquisition depends on the half life and metabolisation of the radiotracer. For \gls{FDG} about $60$ minutes is given.
                    
                    \item The second type of \gls{PET} scan is a dynamic scan. The acquisition begins before the radiotracer is injected into the patient. The injection of the radiotracer during the acquisition allows for the kinetics of the tracer to be observed and quantified with the use of compartmental modelling~\boxcite{Lammertsma2017}. For example, dynamic PET myocardial perfusion imaging (MPI) used in conjunction with tracer kinetic modeling enables the quantification of myocardial blood flow (MBF). Radiotracers such as rubidium $82$ are particularly indicated for dynamic scans given their short half life~\ref{the_physics_of_pet}.
                \end{itemize}
            
            \subsubsection{Attenuation} \label{attenuation}
                Attenuation is the amount of counts lost from annihilation to detection by the scanner while the photons are traversing through the body of the patient. Attenuation can amount to a loss of up to $95$ percent of the total initial signal and can cause increased issues in larger bariatric patients who quite literally have more matter that the photons have to pass through increasing the likelihood that they will be scattered or otherwise attenuated~\boxcite{book, Essential2012}.
                
                There are $4$ main ways through which the photon signal can be lost~\boxcite{book2}, these are in ascending order of magnitude:
                
                \begin{itemize}
                    \item Pair production can be thought of as the inverse process compared to annihilation (as discussed above). This is where a subatomic particle and its antiparticle, such as a electron and a positron, are created from a fundamental particle, such as a photon, usually in close proximity to an atomic nucleus. However, because of conservation of energy a photon would need to be of at least $1.022$ \gls{MeV} which is not generally possible for photons created through electron positron annihilation.
                    
                    \item Rayleigh scattering is the elastic scattering of photons without loss of significant energy by particles which are much smaller than the wavelength of the photon. A common example of Rayleigh scattering is the scattering of sunlight in the atmosphere which is causes the blue colour of the sky during the day and the red colour of the sky at sunset. Because the wavelength of gamma photons is comparably small, compared to most particles, the probability of Rayleigh scattering occurring is neglagable and thus it is normally ignored in \gls{PET}.
                    
                    \item Absorption through the photoelectric effect is the process through which the high energy gamma photon hits and transfers its energy to a material causing the emission of lower energy electrons. The likelihood of the photoelectric effect is inversely proportional to the cube of the photon energy; it also increases as the atomic number of the attenuating material increases. In the matter of the patient the photoelectric effect is most prevalent at photon energies below $100$ \gls{KeV} and as such the probability of the photoelectric effect occurring here is minimal~\boxcite{book}. Photoelectric effect occurs mostly in the detectors of the scanner.
                    
                    \item Compton scattering comprises the majority of interactions between the photon and matter, it occurs where the photon interacts with an electron in a close by atom. The recoiling electron causes the photon to be deflected along another path transferring energy from photon to electron. Compton scattering is also known as incoherent scattering because of its effect on trajectory of the photon. The probability of Compton scattering is indirectly proportional to the energy of the photon~\boxcite{book}.
                \end{itemize}
                
                The relationship between the attenuation of the signal and the material through which it is travelling is given by the Beer-Lambert law. Given $I_0$ incident photons travelling across a path of length $D$, the number of non-attenuated photons $\rmI_{\rmD}$ is given by:
                 
                \begin{equation} \label{eq:beer_lambert_law}
                    \rmI_{\rmD}=\rmI_{0} \cdot \exp \Bigg(\int_{\rmD}-\mu_E(r)\rmd r\Bigg)
                \end{equation}

                \noindent where $\mu_E(r)$ is the attenuation coefficient of the media crossed by the photons of energy $E$.
        
        \subsection{Data acquisition} \label{data_acquisition}
            \blindtext
            % write a general intro on the scanner structure, and the type of events you detect (true scatter randoms etc)
            
            \subsubsection{Photon Detection} \label{photon_detection}
                \blindtext
            
            \subsubsection{Coincidence Processing} \label{coincidence_processing}
                \blindtext
                % write stuff about coincidence processing , give some info on TOF (or separate section for it, i don't know)
            
            \subsubsection{Time of Flight PET} \label{tof_pet}
                \blindtext
            
        \subsection{Attenuation Correction} \label{attenuation_correction}
            \blindtext
            
                \subsubsection{Combined PET/CT} \label{combined_pet_ct}
                    \blindtext
    
    \longsection{PET Image Reconstruction}{pet_image_reconstruction}
        \blindtext
        
        \subsection{Analytic Image Reconstruction} \label{analytic_image_reconstruction}
            \blindtext
            % mention briefely FBP ?
        
        \subsection{Iterative Image Reconstruction} \label{iterative_image_reconstruction}
            \blindtext
            
            \subsubsection{Objective Function} \label{iterative_image_reconstruction_objective_function}
                \blindtext
                
            \subsubsection{Optimiser} \label{iterative_image_reconstruction_optimiser}
                \blindtext
    
    \longsection{Respiratory Motion in PET}{respiratory_motion_in_pet}
        \blindtext
        % general intro
        
        \subsection{Respiratory Motion Artefacts} \label{respiratory_motion_artefacts}
            \blindtext
            % have a look at https://www.sciencedirect.com/science/article/pii/S0001299808000214?via%3Dihub
            
        \subsection{Respiratory Motion Challenges in Combined PET/CT Imaging} \label{respiratory_motion_challenges_in_combined_pet_ct_imaging}
            \blindtext
            
            \subsubsection{Attenuation Correction Techniques in Respiratory Motion Correction} \label{attenuation_correction_techniques_in_respiratory_motion_correction}
                \blindtext
    
    \longsection{Motion Correction for PET}{motion_correction_for_pet}
        \blindtext
    
        \subsection{Image Registration} \label{image_registration}
            \blindtext
            
            \subsubsection{Objective Function} \label{image_registration_objective_function}
                \blindtext
                
            \subsubsection{Optimiser} \label{image_registration_optimiser}
                \blindtext
            
        \subsection{Respiratory Signal Detection} \label{respiratory_signal_detection}
            \blindtext
            
            \subsubsection{External Devices} \label{external_devices}
                \blindtext
                
            \subsubsection{Data Driven} \label{data_driven}
                \blindtext
                
        \subsection{Respiratory Gating} \label{respiratory_gating}
            \blindtext
            % brief intro
            
        \subsection{Motion Modelling} \label{motion_modelling}
            \blindtext
        
        \subsection{Applying Motion Correction} \label{applying_motion_correction}
            \blindtext
            
            \subsubsection{Post Image Reconstruction Motion Correction} \label{post_image_reconstruction_motion_correction}
                \blindtext
                
            \subsubsection{Iterative Motion Corrected Image Reconstruction} \label{iterative_motion_corrected_image_reconstruction}
                \blindtext
    
    \longsection{Machine Learning for PET}{machine_learning_for_pet}
        \blindtext
        
        \subsection{Machine Learning Concepts} \label{machine_learning_concepts}
            \blindtext
        
        \subsection{Machine Learning for PET Image Reconstruction} \label{machine_learning_for_pet_image_reconstruction}
            \blindtext
            % might want to mention this?
        
        \subsection{Machine Learning for Motion Correction} \label{machine_learning_for_motion_correction}
            \blindtext
